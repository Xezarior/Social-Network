\documentclass[10pt,a4paper]{article}

\usepackage[utf8]{inputenc}
\usepackage[french, german]{babel}

\author{Xezarior}
\title{Plague Inc: Evolved \\ \textbf{Social Network (von www.indiemag.fr)}\\ Deutsche Übersetzung}

\usepackage[left=1.6cm,right=1.6cm,top=1.8cm,bottom=2cm]{geometry} 		

\begin{document}
\maketitle


\section{Description -\emph{Beschreibung} - 300Zeichen}

Serez-vous capable de créer le réseau social ultime ?\\
\\
Vous vous réveillez un beau jour avec un grand projet en tête, connecter les gens entre eux via l'internet. Seulement, tout ne se passera pas comme prévu...\\

\textit{Werdet ihr in der Lage sein, das ultimative soziale Netzwerk zu erschaffen ? \\
\\
Ihr erwacht eines Morgens mit dem Willen, gro\ss es zu erreichen : die Menschen miteinander durch das Internet zu verbinden. Doch leider läuft nicht alles wie geplant...}



\section{Popup 1 - 300Zeichen}

Vous vous réveillez un beau jour avec un grand projet en tête, connecter les gens entre eux via l'internet. Seulement, tout ne se passera pas comme prévu...\\

\textit{Ihr erwacht eines Morgens mit dem Willen, gro\ss es zu erreichen : die Menschen miteinander durch das Internet zu verbinden. Doch leider läuft nicht alles wie geplant...}


\section{Popup 2 - 300Zeichen}

Dominer le monde, et ensuite ?\\
\\
Jusqu'où serez-vous capable d'étendre votre réseau sans vous faire stopper ? Qu'allez vous faire une fois au sommet ? C'est à vous d'en décider.\\
\\
(Scénario proposé par www.indiemag.fr)\\

\textit{Die Welt erobern... und dann ?\\ \\ Wie weit werdet ihr es schaffen euer Netzwerk zu verbreiten, ohne aufgehalten zu werden ? Was werdet ihr tun, habt einmal den Höhepunkt erreicht ? Die Entscheidung liegt bei euch.\\ \\ (Szenario angeboten von www.indiemag.fr)}


\section{Victoire alternative -\emph{Alternatives Ende}}

Après avoir causé tant de tort, vous avez finalement décidé d'expier vos fautes. Toute votre fortune est aujourd'hui destinée à rendre le monde meilleur. \\ \\ (Scénario proposé par www.indiemag.fr)\\

\textit{Nachdem Sie Schaden zugefügt haben, habt Ihr euch schliesslich entschlossen, eure Schuld Abzubü\ss en. Ihr gesamtes Vermögen wird heute dazu verwendet, diese Welt besser zu gestalten. \\ \\ (Szenario angeboten von www.indiemag.fr)}


\section{Message de fin -\emph{Spielende}}

Vous avez (finalement) décidé de sauver l'humanité !\\

\textit{Ihr habt euch (letztlich) dazu entschieden, die Menschheit zu retten !}


\section{Traits -\emph{Upgrades (?)} - 30+165 Zeichen}

\paragraph{Démarchons, démarchons...} I : Développer dans son coin, c'est une chose. Mais au bout d'un moment, il faut aussi songer à vous faire connaître en frappant à quelques portes.\\

\textit{Werben, werben...}I\\
\textit{In seiner Ecke zu arbeiten ist ja schön und gut. Aber irgendwann sollte man auch daran denken, mal an ein paar Türen zu klopfen.}


\paragraph{Démarchons, démarchons...} II : Grâce à votre réseau de contacts, vous voila prêt à établir de solides relations avec des acteurs influents du net.\\

\textit{Werben, werben...}II\\
\textit{Danke ihrer Kontakte seid ihr nun in der Lage, mit den einflussreichen Netzakteuren Solide Beziehungen aufzubauen.}


\paragraph{Salons et conférences} : Du CES à l'E3, rien ne vous fait peur ! Vous avez la possibilité de gérer l'événementiel comme un(e) chef et vous êtes prêt à y mettre le prix.\\

\textit{Fachmessen und Konferenzen}\\
\textit{Keine Messe ist vor euch sicher ! Meisterhaft leitet Ihr Werbeveranstaltungen und seid bereit, den Preis dafür zu zahlen.}


\paragraph{Par delà les frontières} I : Il est grand temps de s'exporter ! Vous avez maintenant les ressources humaines et financières nécessaires pour voyager et faire connaître votre site à l'étranger.\\

\textit{Über die Grenzen hinaus}I\\
\textit{Es wird höchste Zeit, seine Arbeit zu exportieren ! Ihr habt nun die nötigen Ressourcen, euer Projekt im Ausland zu verbreiten.}


\paragraph{Par dela les frontières} II : Le monde doit entendre parler de vous. Vous avez dorénavant les moyens de prendre l'avion et le bateau comme vous preniez le Metro !\\

\textit{Über die Grenzen hinaus}II\\
\textit{Die Welt soll von euch hören. Ihr seid nun per Flugzeug und Schiff so unterwegs, wie früher mit der U-Bahn.}


\paragraph{Par dela les frontières} III : Bien assis dans votre fauteil, vous pouvez vous permettre d'envoyer vos équipes de com' pour faire connaître votre site !\\

\textit{Über die Grenzen hinaus}III\\
\textit{Gemütlich in eurem Sessel sitzend, schickt Ihr eure Werbeteams, um eure Internetseite berühmt zu machen.}


\paragraph{Bouche à oreille I} : Pour faire connaître votre site, le bouche à oreille, c'est la base !\\

\textit{Mundpropaganda I}\\
\textit{Seine Seite bekannt zu machen, fängt schon beim Hörensagen an !}


\paragraph{Bouche à oreille II} : Quelques utilisateurs commencent à vous aimer plus que de raison et ils veulent vous aider. Ils ne parlent plus que de votre site et vous soutiennent à jamais.\\

\textit{Mundpropaganda II}\\
\textit{Manche Nutzer entwickeln einen unvernünftig starken Gefallen an euch und wollen helfen. Sie werden euch für immer unterstützen.}


\paragraph{Conditions extrêmes I} : Vous offrez de l'équipement à vos agents de communication afin qu'ils puissent travailler dans de bonnes conditions, peu importe le climat.\\

\textit{Extreme Bedingungen I}\\
\textit{Sie rüsten ihre Werbeagenten so aus, dass sie immer in guten Bedingungen arbeiten können. Das Klima spielt dabei keine Rolle.}


\paragraph{Conditions extrêmes II} : Grâce aux nouvelles technologies, même le grand froid et le climat aride ne fait plus peur à vos employés !\\

\textit{Extreme Bedingungen II}\\
\textit{Dank der neuen Technologien haben Ihre Angestellten selbst vor eisigem Frost und kargem Klima keine Angst mehr.}


\paragraph{Publicité I} : Avec un peu de budget, vous allez enfin pouvoir vous faire connaître par la publicité WEB.\\
\textit{Werbung I}\\
\textit{Mit etwas Budget könnt Ihr endlich durch Internet-Werbung bekannt werden.}


\paragraph{Publicité II} : Les bloqueurs de pubs vous freinent, ne serait-il pas temps de passer un accord avec eux ?\\
\textit{Werbung II}\\
\textit{Die Werbeblocker bremsen euch, wäre es nicht an der Zeit, ein Abkommen mit ihnen zu treffen ?}


\paragraph{Publicité III} : Des partenariats avec les grandes chaînes de Télévision vous permettent de devenir (encore plus) populaire.\\
\textit{Werbung III}\\
\textit{Partnerschaften mit gro\ss en Fernsehsendern lassen Sie (noch) populärer werden}


\paragraph{Images Subliminales} : Grâce à une technologie avancée, vous parvenez à manipuler vos lecteurs à leur insu pour qu'ils vous fassent connaître.\\
\textit{Unterschwellige Bilder}\\
\textit{Dank fortgeschrittener Technologien, beeinflussen Sie nun ihr Benutzer so, dass sie gegen ihren Willen für Sie Werben}


\paragraph{Le net pour tous I} : Vos équipes installent internet dans les pays pauvres. Après tout, ce sont aussi des utilisateurs potentiels, non ?\\
\textit{Ein Netz für alle I}\\
\textit{Ihre Teams richten Internetanschlüsse in Armen Ländern ein. Immerhin sind diese auch potentielle Nutzer, nicht wahr?}


\paragraph{Le net pour tous II} : Envie d'accélérer encore un peu les choses ? Il va falloir investir un peu plus..\\
\textit{Ein Netz für alle II}\\
\textit{Lust, das ganze noch etwas mehr zu beschleunigen? Da werden Sie wohl noch etwas mehr investieren müssen...}


\paragraph{Ballons Internet I} : "Comment ? Internet n'est pas partout sur Terre ? Aucun soucis, nous leur amènerons sur un plateau."\\
\textit{Internet-Ballons I}\\
\textit{Wie bitte? Das Internet ist nicht von überall aus erreichbar? Kein Problem, bringen es ihnen auf einem Silbertablett.}


\paragraph{Ballons Internet II} : Grâce à une nouvelle technologie, vos ballons sont encore plus performants et permettent d'offrir internet à la totalité de la Terre.\\
\textit{Internet-Ballons II}\\
\textit{Dank der Fortschritte, sind Ihre Luftballons nun noch Leistungsfähiger, und vernetzen nun die gesamte Erdoberfläche.}


\paragraph{Fidélité I} : Votre communauté s'étend, et vous aime d'amour. Il consultent régulièrement votre site et y reviennent volontiers.\\
\textit{Treue I}\\
\textit{Ihre Gemeinschaft erweitert sich, und liebt Sie wörtlich. Sie sehen Ihre Seite regelmäßig ein, und besuchen sie gerne erneut.}


\paragraph{Fidélité II} : Les utilisateurs ne peuvent plus se passer de votre réseau social. Mais que leur avez-vous fait ?\\
\textit{Treue II}\\
\textit{Ihre Nutzer können nicht mehr auf euer Soziales Netzwerk verzichten. Was habt Ihr ihnen bloß angetan ?}


\paragraph{Addiction I} : Les autorités s'inquiètent. Vos utilisateurs ne sortent plus de chez eux, ils passent leurs journées sur votre site au détriment de leur santé.\\
\textit{Sucht I}\\
\textit{Die Obrigkeiten beunruhigen sich. Eure Nutzer ihre Tage auf eurer Webseite und vernachlässigen ihre Gesundheit.}


\paragraph{Addiction II} : L'heure est grave ! Tous vos lecteurs ont arrêté de vivre normalement et n'ont que votre site en tête au point de se laisser mourir de faim et de soif.\\
\textit{Sucht II}\\
\textit{Die Situation wird kritisch ! Sämtliche Ihrer Leser denken nur noch an eure Webseite und vergessen, zu Essen und zu trinken.}


\paragraph{Concurrence déloyale} : Ah, la jalousie ! Les ennuis débarquent, certains vous veulent du mal et dénoncent vos pratiques douteuses. "Il n'y a pas de mauvaise pub"\\
\textit{Unlauterer Wettbewerb}\\
\textit{Ha, Neid ! Ärger kündigt sich an, manche wollen Ihnen böses und finden Ihre Mittel fragwürdig. "Schlechte Werbung gibt es nicht"}


\paragraph{Concurrence déloyale II} : Les grands groupes se réveillent et décident de vous enterrer ! Ils sont prêts à tout, même à tuer vos lecteurs, pour que l'on vous empêche d'exister sur le Web.\\
\textit{Unlauterer Wettbewerb}\\
\textit{Die mächtigen Gruppen sind nun zu allem bereit, euch los zu werden. Sogar, eure Nutzer zu töten um eure Existenz zu verhindern.}


\paragraph{Concurrence déloyale III} : On ne rigole plus. Grâce à un Virus informatique très performant, les sociétés concurrentes parviennent à tuer tous les visiteurs de votre site.\\
\textit{Unlauterer Wettbewerb}\\
\textit{Schluss mit Lustig. Dank eines mächtigen Computer-Virus, schaffen es Ihre Konkurrenten, sämtliche eurer Nutzer zu töten.}


\paragraph{C'était mieux avant} : Votre site grandit, il change, devient commercial, les gens se plaignent et certains en viennent à commettre l'irréparable.\\
\textit{Früher war alles besser}\\
\textit{Eure Seite wächst, verändert sich, wird kommerziell, die Leute beschweren sich und manche begehen das unwiderrufliche.}


\paragraph{C'était mieux avant II} : Déprime massive en perspective. Malgré la venue d'un nouveau public, l'évolution déplait aux plus fidèles qui se suicident en masse.\\
\textit{Früher war alles besser II}\\
\textit{Massive Depressionen sind zu erwarten. Trotz des neuen Publikums missfallen die Änderungen den treuesten Nutzern, die ihr daraufhin ihr Leben beenden.}


\paragraph{C'était mieux avant III} : Mais... Qu... Qu'êtes vous devenu ? Vous contrôlez le monde... mais le monde vous DETESTE ! À quoi bon continuer de vivre comme ça ?\\
\textit{Früher war alles besser III}\\
\textit{Aber... Wa... Was ist aus Ihnen geworden? Sie beherrschen die Welt... doch die Welt HASST euch ! Wozu sollte man so weiterleben ?}


\paragraph{Crise d’épilepsie} : Jouer avec le cerveau de vos utilisateurs, c'est mal. Ne vous étonnez donc pas de leur état de santé.\\
\textit{Epileptischer Anfall}\\
\textit{Mit den Gehirnen ihrer Nutzer zu spielen ist falsch. Also wundert euch nicht über deren gesundheitlichen Zustand.}


\paragraph{Décimez-les tous ! I} : Vous lancez \textbf{une la!!!!!} mode des \textbf{"puces ADN"} connectées permettant à quiconque de naviguer sur votre site par la pensée. Mais à quels fins ?\\
\textit{Vernichtet sie alle ! I}\\
\textit{Sie bringen die Mode der DNA-Mikrochips auf den Markt, die Jedermann die Möglichkeit geben per Gedanken auf Ihrer Website zu surfen. Doch mit welchem Ziel ?}


\paragraph{Décimez-les tous ! II} : Vous l'aviez bien caché. Les puces ADN peuvent maintenant relacher une substance mortelle dans l'organisme de l'hôte. Etaient-ils au courant ?\\
\textit{Vernichtet sie alle II}\\
\textit{Ihr hattet es gut versteckt. Die Mikrochips können nun eine tödliche Substanz in den Organismen der Träger freisetzen. Wussten diese Bescheid ?}


\paragraph{Pour la bonne cause...} : Vous donnez chaque mois un peu de votre argent à des oeuvres charitatives. Même la bonne image peut avoir un prix ! (Ralentit les recherches)\\
\textit{Für einen guten Zweck...}\\
\textit{Sie geben jeden Monat ein bisschen Geld an karitative Vereine. Sogar das gute Image hat seinen Preis ! (Verlangsamt den Forschungsfortschritt)}


\paragraph{Pour la bonne cause... II} : Lors d'événements, vous ne manquez pas de vous faire remarquer en alignant un gros chèque pour laver votre honneur. (Ralentit les recherches)\\
\textit{Für einen guten Zweck II}\\
\textit{Bei Ereignissen verfehlen Sie keine Gelegenheit, positiv aufzufallen und kaufen Ihre Ehre mit einem Großen Scheck wieder sauber. (Verlangsamt den Forschungsfortschritt)}


\paragraph{Pour la bonne cause... III} : En vous regardant dans le miroir, ce matin-là, vous ne vous reconnaissez plus. Vous décidez de tout plaquer pour faire de cette Terre un monde meilleur.\\
\textit{Für einen guten Zweck III}\\
\textit{Als Sie sich an diesem Morgen im Spiegel erblicken, erkennen Sie sich nicht mehr wieder. Ihr gebt alles auf, und wollt aus dieser Erde eine bessere Welt machen.}


\paragraph{Les finances, ça compte I} : Grâce à une petite régie publicitaire et quelques prestations modestes, vous parvenez à faire gonfler votre chiffre d'affaire.\\
\textit{Finanzen sind, was zählt I}\\
\textit{Dank eines kleinen Werbevermarkters und ein paar bescheidenen Leistungen nimmt Ihr Umsatz langsam zu.}


\paragraph{Les finances, ça compte II} : Goodies, régies influentes, prestations de grande envergure, vous êtes sur tous les fronts ! Les investisseurs vous aiment.\\
\textit{Finanzen sind, was zählt II}\\
\textit{Goodies, einflussreiche Werbevermarkter, grosse Werbeleistungen, Ihr kämpft an allen Fronten ! Die Investoren lieben euch.}


\paragraph{Les finances, ça compte III} : Grâce à vos vilaines combines, vous êtes riche... Très riche ! Votre site est maintenant côté en bourse et l'argent n'est plus un soucis.\\
\textit{Finanzen sind, was zählt III}\\
\textit{Dank Ihrer gemeinen Tricksereien sind sie reich geworden... sehr reich ! Eure Website ist nun an der Börse und Geld stellt kein Problem mehr da.}


\paragraph{Référencement }: C'est quoi un tag ?\\
\textit{Referenzierung}\\
\textit{Was ist eigentlich ein tag?}


\paragraph{Référencement II} : Vous créez votre propre moteur de recherche : Boogle ! Tous les liens mènent à vous et la recherche de solutions pour vous contrer s'en trouve très ralentie.\\
\textit{Referenzierung II}\\
\textit{Sie begründen Ihre eigene Suchmaschine : Boogle ! Alle Links führen zu Ihnen und die Suche nach Lösungen, Sie zu umgehen wird stark verlangsamt.}


\paragraph{Petit site deviendra grand }: Il est beau, fonctionnel, comme vous l'imaginiez... Ou pas ! C'est un véritable désastre, un sous-blog, mais vous en êtes (pour le moment) content...\\
\textit{Kleine Website wir mal groß}\\
\textit{Sie ist schön, funktioniert, genau wie in Ihren Vorstellungen... oder auch nicht! Eigentlich ist es ein Desaster aber (für den Moment) seid Ihr damit zufrieden.}


\paragraph{Petit site deviendra grand II} : Après avoir lu "Le développement pour les nuls", vous faites vos premières armes en la matière et arrivez à améliorer votre bébé de façon significative.\\
\textit{Kleine Website wir mal groß II}\\
\textit{Nachdem sie "Enwticklung für Dummies" gelesen haben, bauen Sie Ihre ersten Waffen und bringen Ihr Baby um einiges weiter.}


\paragraph{Petit site deviendra grand III} : Au vu de votre chiffre d'affaire, il serait criminel de continuer à gérer la technique de votre site. Laissez faire les professionnels !\\
\textit{Kleine Website wir mal groß III}\\
\textit{Wenn man Ihren Umsatz betrachtet, wäre es fast Kriminell weiterhin selber die Technik Ihrer Website zu verwalten. Lassen sie lieber die Profis ran.}


\paragraph{Petit site deviendra grand IV} : Vous avez perdu la tête ? La dépression vous guette... Vous décidez de conquérir le monde pour... détruire l'humanité ?\\
\textit{kleine Website wir mal groß IV}\\
\textit{Habt Ihr euren Verstand verloren ? Depressionen zeigen erste Anzeichen... Sie entscheiden sich dafür, die Welt zu erobern um... die Menschheit zu vernichten ?}


\paragraph{La vidéo, c'est l'avenir ?} : Les internautes ne se content plus d'images, il veulent partager des vidéos; C'est pourquoi, vous concluez d'un partenariat avec une plateforme dédiée à ça : WeTube\\
\textit{Sind Videos die Zukunft ?}\\
\textit{Die Internetbenutzer wollen mehr, als nur Bilder : sie möchten Videos teilen können. Daher schließen sie ein Abkommen mit einer zweckgebundenen Plattform ab : WeTube}


\paragraph{Social Network} : Vous devenez un réseau social. Capable de gouverner le monde, mais aussi de générer l'addiction chez vos utilisateurs.\\
\textit{Sie werden zu einem sozialen Netzwerk. In der Lage, die Welt zu regieren, und Ihre Nutzer Süchtig zu machen.}



\section{Texte de Description -\emph{Beschreibung(stext)}}

Hello !

Parce qu'il n'y a pas que les maladies et le virus qui peuvent coller au gameplay de Plague, j'ai décidé de tenter l'expérience de transposer les mécaniques du jeu à la propagation d'un site web de type "réseau social". Je réclame toute votre indulgence, malgré les heures que j'ai pu passer sur ce scénario, il reste très modeste et n'a pas la prétention de révolutionner le jeu.\\
\\
En revanche, beaucoup de points ont été réfléchis et voici une description de ce qui vous attend :\\
- Élaborez la meilleure stratégie possible pour étendre votre site web. 1 infecté = 1 utilisateur\\
- Lorsque votre site web devient trop gros, la jalousie des autres et le mécontentement de vos utilisateurs peuvent vous causer du tort\\
- La "recherche" représente ici votre échec. À 100\%, votre réseau est détruit et vous êtes stoppé.\\
- Allez-vous craquer et détruire l'humanité, ou bien la sauver ? Oui, ce scénario comporte une fin optionnelle !\\
\\
Malgré les nombreuses limitations de l'éditeur, j'ai souhaité travailler un maximum les interfaces en écrivant avec soin toutes les descriptions des traits (transmission, symptômes, capacités) pour assurer une cohérence à l'ensemble. De plus, pour coller au thème, presque toutes les images utilisées ont été créées pour l'occasion afin de facilité la compréhension globale du scénario. Et puis bon, qu'on se le dise, je n'aime pas faire les choses à moitié.\\

\textit{Hallo !}\\
\\
\textit{Das es nicht nur Krankheiten und Viren gibt, die zu Plague's Gameplay passen, habe ich mich an den Versuch gewagt, die Spielmechanismen auf die Verbreitung einer Sozialen Netzwerkes zu übertragen. Ich erbitte trotz der vielen Stunden, die ich mit diesem Szenario verbracht habe, eure gesamte Nachsichtigkeit, da es sehr bescheiden bleibt und nicht danach strebt, das Spiel zu revolutionieren.}\\
\\
\textit{Allerdings, wurden viele Punkte gut durchdacht. Hier eine Beschreibung dessen, was euch erwartet :}\\
\textit{- Erdenken Sie die bestmögliche Strategie, Ihre Website zu verbreiten. 1 Infizierter = 1 Nutzer}\\
\textit{- Wenn Ihre Webseite zu groß wird können der Neid anderer und die Unzufriedenheit der Nutzer Ihnen schaden.}\\
\textit{- Die "Forschung" entspricht hier eurem Versagen. Bei 100\% wird Ihr Netzewerk zerstört und Ihr werdet aufgehalten.}\\
\textit{Werdet Ihr schwächeln, und die Menschheit zerstören, oder sie retten ? Ja, dieses Szenario bietet ein alternatives Ende !}\\
\\
\textit{Trotz aller Einschränkungen des Editors, wollte die Benutzeroberflächen weit möglichst bearbeiten und habe mit Mühe alle Beschreibungen der Upgrades (Verbreitung, Sylptôme, Fähigkeiten) geschrieben, um das Ganze zusammenhängend zu gestalten. Desweiteren, um bei dem Thema zu bleiben, wurden fast alle Bilder genau für diesen Zweck erstellt, um insgesamt die Verständlichkeit des Szenarios zu verbessern. Und ja, jetzt können wir es sagen : ich mache ungern etwas nur zur Hälfte.}










\end{document}